\section{Locality-Sensitive Hashing}

\acrfull{lsh} is one of the methods that tackle the problem of the approximate
nearest neighbor search, its main idea consists of using the hashing
\footnote{It is the process of transforming an object into a relatively shorter
fixed-length value that we call key, and it is used to make the search operation
faster.} techniques to map the near data points to the same buckets \footnote{A
bucket is a group of elements that share the same hash value} with high
probability, and map the distant ones to the same buckets with low probability.
The intuition behind it is to apply a number of hashing functions on the data
points to get their representations in a new latent space, and the goal with
these hashing functions is to have nearby points in the original space close to
each other in the latent space.


\subsubsection{Locality-Sensitive hashing and Classical hashing}
In classical hashing or hash-based search the idea is to load a set of elements
in a hashing structure that has multiple entries and to have a hash function
that maps every element from the set to one of the entries. The goal behind this
choice is to speed up the search process, which can be guaranteed by minimizing
the collisions between the set elements.

On the other hand, in \acrlong{lsh}, the idea is to map data points to different
\glspl{bucket} in a way that the nearby elements fall in the same buckets with high
probability. Unlike classical hashing where the goal is to minimize collisions,
in \acrshort{lsh} we look to maximize them.
