\Acrfull{ann} is a reformulated version of the \Acrlong{nns} where the solution
is found by the use of an index that gives a latent representation of the
original data points, the intuition is to divide the data points into subsets
(\gls{bucket}) of data points according to this new latent representation, and to
select one subset do the search on it. The loss of information between the
original data points and their latent representation makes the solution not
guaranteed to be exact.

The methods implementing the idea of \acrshort{ann} can be classified under four
major families:
\begin{itemize}
      \item Hashing based algorithms: The idea behind this family of methods is
            to apply a set of hashing functions on the data points and to
            consider their results as the new representation (in a lower
            dimensional space) of their original points.
      \item Quantization based algorithms: The idea is to decompose the space
            into a Cartesian product of low dimensional subspaces and to quantize each
            subspace separately. \citep{jegou_pqfnns_2011}
      \item Tree based algorithms: The general idea is to form a tree where the
            root node represents all the dataset, and to start splitting this data to
            two halves by a hyper-plane orthogonal to a chosen dimension at a threshold
            value. \citep{silpanan_2008}
      \item Graph based algorithms: The idea is to represent the data as a graph
            where the nodes are the data points, and an edge between two nodes
            defines a neighbor-relationship between their respective points.
            \citep{ann_mengzaho_2021}
\end{itemize}

Hashing based algorithms, Quantization based algorithms, Tree
based algorithms, and Graph based algorithms.\citep{ann_mengzaho_2021}.

In this present work we explore the hashing based algorithms, or \acrfull{lsh}
in order to see how they can be suitable for the problem of detecting
similarities between datasets.
