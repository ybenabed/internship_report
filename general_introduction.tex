\chapter*{Introduction}
\addcontentsline{toc}{chapter}{Introduction}

As data continues to grow, it is becoming increasingly critical for
organizations to be able to easily manage and sort their datasets, whether they
come from databases, files or other sources. For this purpose, Talend already
offers \href{https://www.talend.com/products/data-inventory/}{Talend Data
Inventory}, a tool that makes it easy to find and understand your data with
powerful and versatile search, provenance and dataset overview. It also helps to
identify data silos in all data sources and eliminates them with reusable and
shareable data assets.

In order to continuously improve the Talend Data Inventory quality of service,  
Talend Lab team is interested in integrating a tool that can detect the
similarities between datasets, a functionality that can be used in different use
cases, the most relevant one is at the moment when the user introduces a new
dataset, we want to find the ones that are similar to it, and by providing this
information to the user we can propose some actions he can do, he can for
example compare and merge the similar datasets or update the existing one with
new records. Another use case is to assign a tag to a dataset according to the
tags of the ones that are similar to it, for example if we find that a dataset
is similar to two dataset with the tag "\textit{finance}", we can suggest this
same tag to the user.

In this work, we tackle the problem of detecting similarities between dataset as
a \Acrfull{nns} problem, we explore the state of the art with focusing on
\Acrfull{lsh} methods and we study their different versions and the metric they
consider.

We present in the second place the solution built during this internship, in
which we use a method of the \acrshort{lsh} family called Cross-Polytope, our
goal is to scale its idea that was initiated to work at row level and get it
work at the column level, we propose for this a set of preprocessing methods and
modifications. We show with this solution the different tests and experiments we
have done to get the suitable configuration and parameters that makes our
solution return better results.

We also show the steps of implementing the solution in the realization section,
where we talk about the technologies and frameworks used, and also we present
the architecture that we use to implement a proof of concept.

\section*{Organization of the report}
\addcontentsline{toc}{section}{Organization of the report}

This report is organized in four chapters:

\subsection*{Chapter 1: State of the art}
In this chapter we define the \acrfull{nns} problem and its approximated
version, then we present one of the methods that tackles it which is
\acrfull{lsh} family and how to formulate it. We also talk about the metrics
that are used generally in data science problems and their estimations with
\acrshort{lsh}. We end this chapter by presenting the common steps between all
\acrshort{lsh} methods and we present some of them.

\subsection*{Chapter 2: Conception}
This chapter presents the logic of our solution and its conception, we start it
by presenting its two parts, the indexation and the similarity detection. And we
end the section by presenting the class diagram.


\subsection*{Chapter 3: Realization}
In this chapter we describe the use case we considered to build the proof of
concept, and then we present the tools and frameworks we used for the
implementation, we present also the architecture considered for building the
demonstration.

\subsection*{Chapter 4: Experiments and results}
The last chapter is about the experiments and tests completed to measure the
accuracy of our solution and also to tune the hyper-parameters of the used
\acrfull{lsh} method. 


\section*{Presentation of the company}
Talend is a French company founded in 2006 providing open source solution of
data integration and management. It is a leader of its industry. Talend's
solutions are the most widely used and deployed in the world today.

\subsection*{Talend products}
We list in this paragraph some of the products provided by Talend:
\subsubsection*{Talend Data Fabric}
\href{https://www.talend.com/products/data-fabric/}{Talend Data Fabric} is a
low-code platform that gives the users with the possibility to unify their data,
to provide a common language of data and to build data trusts.

\subsubsection*{Talend Data Integration}
\href{https://www.talend.com/products/integrate-data/}{Talend Data Integration} 
a tool that integrates on-cloud user data with a secure cloud integration
platform-as-a-service (iPaaS). It provides the user powerful graphical
tools, prebuilt integration templates, and a rich library of components.

\subsubsection*{Talend Data Quality}
\href{https://www.talend.com/products/data-quality/}{Talend Data Quality} is an
integral part of Talend Data Fabric, it has the potential to reduce costs,
increase revenue and improve performance by making reliable data available in
any type of integration. This product profiles, cleans, and masks data in real
time. It also suggests to the user some recommendations powered by machine
learning for addressing data quality issues.





% Nearest neighbor search is a topic that occupied a lot of interest in the
% machine learning search field as its exhaustive solutions require a lot of time
% and resources to find the solution. Due to the complexity in time of the
% exhaustive solutions when the number of points and their dimension get bigger
% (curse of dimensionality), the exact nearest neighbor problem is then
% reformulated into (\Acrfull{ann}).

% \acrfull{lsh} is one of the methods that tackles the problem of the approximate
% nearest neighbor search, it consists of using the hashing technique to map the
% near data points to the same buckets with high probability, and map the distant
% one to the same buckets with low probability.
