\chapter*{Introduction}

As data continues to grow, it is becoming increasingly critical for
organizations to be able to easily manage and sort their datasets, whether they
come from databases, files or other sources. For this purpose, Talend already
offers \href{https://www.talend.com/products/data-inventory/}{Talend Data
Inventory}, a tool that makes it easy to find and understand your data with
powerful and versatile search, provenance and dataset overview. It also helps to
identify data silos in all data sources and eliminates them with reusable and
shareable data assets.

In order to continuously improve the Talend Data Inventory quality of service,
Talend Lab team is interested in integrating a tool that can detect the
similarities between datasets, a functionality that can be used in different use
cases, the most relevant one is at the moment when the user introduces a new
dataset, we want to find the ones that are similar to it, and by providing this
information to the user we can propose some actions he can do, he can for
example compare and merge the similar datasets or update the existing one with
new records. Another use case is to assign a tag to a dataset according to the
tags of the ones that are similar to it, for example if we find that a dataset
is similar to two dataset with the tag "\textit{finance}", we can suggest this
same tag to the user.

In this work, we tackle the problem of detecting similarities between dataset as
a \Acrfull{nns} problem, we explore the state of the art with focusing on
\Acrfull{lsh} methods and we study their different versions and the metric they
consider.

We also present the solution built during this internship, in which we use a
method of the \acrshort{lsh} family called Cross-Polytope, our goal is to scale
its idea that was initiated to work at row level and get it work at the column
level.

% Nearest neighbor search is a topic that occupied a lot of interest in the
% machine learning search field as its exhaustive solutions require a lot of time
% and resources to find the solution. Due to the complexity in time of the
% exhaustive solutions when the number of points and their dimension get bigger
% (curse of dimensionality), the exact nearest neighbor problem is then
% reformulated into (\Acrfull{ann}).

% \acrfull{lsh} is one of the methods that tackles the problem of the approximate
% nearest neighbor search, it consists of using the hashing technique to map the
% near data points to the same buckets with high probability, and map the distant
% one to the same buckets with low probability.
