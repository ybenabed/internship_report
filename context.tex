\chapter*{Context}
\addcontentsline{toc}{chapter}{Context}

\section*{Presentation of the company}
\addcontentsline{toc}{section}{Presentation of the company}
Talend is a French company founded in 2006 providing open source solutions of
data integration and management. It is a leader of its industry. Talend's
solutions are the most widely used and deployed in the world today.

\subsection*{Talend products}
\addcontentsline{toc}{subsection}{Talend products}
We list in this paragraph some of the products provided by Talend
\subsubsection*{Talend Data Fabric}
\href{https://www.talend.com/products/data-fabric/}{Talend Data Fabric} is a
low-code platform that gives the users the possibility to unify their data, to
provide a common language of data and to build data trusts.

\subsubsection*{Talend Data Integration}
\href{https://www.talend.com/products/integrate-data/}{Talend Data Integration} 
is a tool that integrates on-cloud user data with a secure cloud integration
platform-as-a-service (iPaaS). It provides the user with powerful graphical
tools, prebuilt integration templates, and a rich library of components.

\subsubsection*{Talend Data Quality}
\href{https://www.talend.com/products/data-quality/}{Talend Data Quality} is an
integral part of Talend Data Fabric, it has the potential to reduce costs,
increase revenue and improve performance by making reliable data available in
any type of integration. This product profiles, cleans, and masks data in real
time. It also suggests to the user some recommendations powered by machine
learning for addressing data quality issues.


\section*{Internship context}
\addcontentsline{toc}{section}{Internship project}
This internship was carried out in the Talend team, part of the research and
development subsidiary at Talend. The role of the Lab team is to research and
develop machine learning themes to be integrated as a service to Talend's
various products or to take maximum advantage of the various cloud resources.

The objectives of the internship are to study the literature related to dataset
classification and implement techniques to efficiently classify any new dataset. 

I have been given some tasks related to this project:

\begin{itemize}
    \item To study the state of the art on \acrfull{lsh} methods and their uses
    in similarity estimation.
    \item Elaborate paper reviews and add content to the section dedicated to
    the project in the Lab's documentation platform.
    \item Understanding of existing implementations and their limitations.
    \item Discuss a solution with the supervisors and implement it.
    \item Build a demonstration application and present it with a use case to
    show its features.
\end{itemize}