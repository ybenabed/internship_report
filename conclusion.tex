\chapter*{Conclusion}
\addcontentsline{toc}{chapter}{Conclusion}

This internship period has been full and rich in terms of acquiring new
experiences and skills. It was an experience that not only allowed me to learn
more about the field of machine learning, but it was also an opportunity to
develop on the organizational side, in terms of work methodology, quality of
exchanges, requirements on deliverables and the importance of software
engineering itself. 


Through this report, I have exposed the work done in the project of my graduation
internship which consisted in doing a state of the art and an implementation to
address the need of having a tool able to detect the similarities between
dataset based on their content with Cross-Polytope which is a \acrfull{lsh}
method.

I started this report with a definition on the \acrfull{nns} problem, how need
it is to scale it to \acrfull{ann} and I exposed the metrics that it can use to
measure the similarities or dissimilarities between data points. I continued by
presenting \acrshort{lsh} family of methods as one of the methods that tackle
\acrshort{nss}, I explained its global approach and finally presented their
different methods related to each one of the previously mentioned metrics.

My solution starts by initiating a set of preprocessing over the dataset, it
first needs to get its columns separated into textual and non-textual column,
embed the textual ones and get a sample from the second one. This solution is
based on indexing the columns after preprocessing with Cross-Polytope LSH, a
method that gives an estimation related to the angular distance. And to get the
similarity measures with this solution, we need to get the buckets of each
column with its indexes, and then compute the similarity between the indexes that share at least one bucket.


\section*{Perspective}
Despite the good results shown by this solution, there is still work to be done
before it goes into production. This work will be mainly to see how to integrate
it as a feature in Talend Data Inventory, and to get the first feedbacks from
the users taking in consideration that they can have datasets in different forms
and with so many types, something that couldn't be guaranteed with the benchmark
we used for the tests.