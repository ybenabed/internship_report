\subsection{Indexation hyper-parameters}
In this second test the goal is to find the best combination of number of the
hashing functions to use in indexation and the number of probes to use in
Multi-Probe LSH. The best combination is the one that gives the index with the
minimum error of estimation of the angular distance. To do this, a list of
values for each of the two variables and a set of 1000 dataset couples have been
set, and the error for a combination of a number of probes and a number of
hashing functions is computed by computing the mean absolute error for each
couple of datasets and computing the mean of error over all the list of couples.

The pseudo-algorithm \ref{alg:crosspolytope_test} describes the steps to
complete this test routine.

\begin{algorithm}[H]
    \SetAlgoLined
    \label{alg:crosspolytope_test}
    \KwIn{num\_probes: List of integer\;
        num\_hashes: List of integer\;
        dataset\_list: List of dataset couples.
    }
    error\_matrix: matrix

    \For{(num\_probe, num\_hash) in (num\_probes, num\_hashes)}
    {
        errors = []

        \For{($dataset_1$, $dataset_2$) in dataset\_list}
        {
            error = similarity\_estimation\_error($dataset_1$, $dataset_2$)
            add\_element(errors, error)
        }
        error\_matrix[num\_probe, num\_hash] = mean(errors)
    }

    \Return error\_matrix
    \caption{Pseudo-algorithm for the similarity estimation test routine.}

\end{algorithm}

The similarity\_estimation\_error function is computed by indexing the datasets
in input with the corresponding configuration and compute the mean absolute
error with the exact similarities.

The results of this test are represented in the following table where the
columns correspond to the number of probe values and the rows correspond to the
number of hashing functions.

\hspace{1cm}

\begin{table}[h]
    \begin{center}
        \begin{tabular}{|c | c | c | c| c| c| c|}
            \hline
            \diagbox[width=10em]{Number of                                    \\hashing func}{Number of\\Probes}
                          & \textbf{3} & \textbf{4} & \textbf{5} &
            \textbf{6}    & \textbf{7} & \textbf{8}                           \\
            \hline

            \textbf{16}   & 0.084150   & 0.091189   & 0.102268   & 0.114099 &
            0.126253      & 0.138592                                          \\
            \hline
            \textbf{32}   & 0.069658   & 0.073694   & 0.084380   & 0.097009 &
            0.110436      & 0.123476                                          \\
            \hline
            \textbf{64}   & 0.060197   & 0.063477   & 0.075861   & 0.090637 &
            0.105167      & 0.119083                                          \\
            \hline
            \textbf{128}  & 0.054985   & 0.056040   & 0.069854   & 0.086120 &
            0.101522      & 0.115687                                          \\
            \hline
            \textbf{256}  & 0.049240   & 0.055169   & 0.072686   & 0.089780 &
            0.105259      & 0.119369                                          \\
            \hline
            \textbf{512}  & 0.047183   & 0.053220   & 0.072063   & 0.089319 &
            0.104874      & 0.119079                                          \\
            \hline
            \textbf{1024} & 0.046043   & 0.052195   & 0.071337   & 0.088481 &
            0.103913      & 0.117931                                          \\
            \hline
        \end{tabular}
    \end{center}
    \caption{Results for the multi-probe hyper-parameters testing}
    \label{table:multiprobe_hyperparameters}
\end{table}



The table \ref{table:multiprobe_hyperparameters} shows the results of this
experiment, from which we can take the following points:
\begin{itemize}
    \item The best configurations gave errors in the interval of $[0.45, 0.55]$.
    \item The results confirms the theoretical result in
          \citep{andoni_practicalsh_2015} saying that by using a multi-probe
          technic we can improve the results of estimation without having to
          increase the number of hashing functions used, for example, the
          difference of estimation error between the case of $256$ hashing
          functions and $1024$ hashing functions is at the range of $0.003$.
\end{itemize}


